\section{Fazit \& Ausblick}

\subsection{Fazit}

Durch die Literaturrecherche zu den technologischen Möglichkeiten der Kontaktaufnahme konnte die Auswahl der erfolgversprechend einsetzbaren Technologien auf die drei Kandidaten QR Codes, NFC Tags und Kurz—URLs eingeschränkt. Bluetooth zeigte sich hauptsächlich wegen der hohen Komplexität und die im betrachteten Einsatzgebiet beschränkten Möglichkeiten als ungeeignet. \ac{BLE} erweitert diese Möglichkeiten und reduziert die Komplexität, so dass Anwendungen möglich werden, die für das Gastgewerbe sehr interessant sein können. Jedoch sind diese Möglichkeiten momentan nicht für kleine Betriebe effizient umsetzbar.

Die Kombination von geringen Kosten, geringer Komplexität und hoher Bekanntheit bzw. Verbreitung sowie die weiteren Ergebnisse der Befragung stützen die gegebene Handlungsempfehlung, lesbare kurze \ac{URL}s, evt. in Kombination mit QR Codes einzusetzen.

\subsection{Ausblick}

Nachdem in dieser Arbeit das Hauptaugenmerk auf den technischen Möglichkeiten liegt, bieten sich für weitere Forschungen die psychologischen Aspekte an. Anhand von Tests könnte untersucht werden, welche der Möglichkeiten am besten konvertiert, also welche Technologie tatsächlich die meisten Kunden zum digitalen Angebot des Betriebs leitet.

Auch bietet es sich an, die verschiedene Punkte im Serviceprozess des Betriebs darauf zu untersuchen, wie gut sich ebendieser für einen Wechsel auf digitale Angebote eignet.

Des weiteren sollten die Möglichkeiten bewertet werden, mit denen die digitale Kommunikation selbst stattfinden kann. Hierzu gehören neben den in dieser Arbeit bereits erwähnten Messenger und proprietäre Smartphone–Apps von einzelnen Betrieben, aber auch weitere soziale Netzwerke wie Instagram, klassische E–Mails bis hin zum Schritt zurück zum klassischen Mailing per Briefpost. 

