\section{Einleitung}
\label{sec:einleitung}

\subsection{Begründung der Problemstellung} %(fold)

Im Februar 2014 nutzen 40,4 Millionen Deutsche ein Smartphone\footnote{vgl. \cite{netzoekonom}}. Dies entspricht einem Anteil von knapp 50\% der Bevölkerung\footnote{vgl. \cite{destatis:bev}}, bzw. durschscnittlich eines in fast jedem deutschen Haushalt\footnote{vgl. \cite{destatis:hh}}. 

Mit dieser stetig steigenden Verbreitung von Smartphones in der Bevölkerung nehmen auch die Möglichkeiten der digitalen Kontaktaufnahme zu. 
Ein stark wachsender Anteil von 27\% der Reisenden informieren sich wärend ihrer Reise über das mobile Internet. Ganze 36\% der Reisendenden  teilen Ihre Reiseerlebnisse im Social Web\footnote{vgl. \cite{reiseanalyse}, Seite 5}. Somit ist das Thema für Hotellerie und Gastronomie ein Zukunftsthema, das auch schon heute relevant ist.

Hierbei konkurrieren verschiedene Technologien miteinander. Allen gemeinsam ist es, dass die Kunden oder Interessenten vor Ort eine kleine Menge\footnote{Üblich sind hier Datenmengen im Bereich von einigen hundert Bytes oder wenigen Kilobytes} an Daten auf ihr Smartphone übermittelt bekommen.

Unternehmen, die diese Möglichkeiten nutzen möchten, brauchen eine verlässliche Basis für die Entscheidung welche Technologie zum Einsatz kommen sollen. Obwohl technisch nichts dagegen spricht, mehrere Technologien gleichzeitig anzubieten, so nehmen sich diese dann doch den Platz und die Aufmerksamkeit des Kunden. In kleinen Betrieben sind schon aufgrund der Umsatzhöhe das Budget sowohl für IT als auch für Marketing beschränkt. Konzeption und Umsetzung werden in diesen Betrieben meist von Mitarbeitern erledigt, die gleichzeitig noch andere Aufgaben zu erfüllen haben. Daher sind auch die Bewertungskriterien entsprechend diesen Erfordernissen zu gestalten. Die Entscheidung zwischen den verschiedenen Technologien ist bei kleinen Betrieben noch bedeutungsvoller als bei großen Konzernen.

Informationstechnisch betrachtet handelt es sich beim betrachteten Vorgang einen \buzz{Medienbruch}. Die impliziert die Existenz eines Quellmediums, einer Schnittstelle und eines Zielmediums.

% (end)
\subsection{Ziele dieser Arbeit} %(fold)

\textbf{Ziel dieses Projektberichts ist es, Handlungsempfehlungen für den Einsatz von Technologien zur digitalen Kontaktaufnahme mit Kunden in kleinen Betrieben des Gastgewerbes zu geben.}

Hierzu werden zunächst im Kapitel~\myref{sec:grundlagen} die für diese Arbeit relevanten Begriffe und Konzepte definiert, und im Kapitel~\myref{sec:tourismusbranche} die Branche kurz beschrieben, bevor im Kapitel~\myref{sec:technologien} eine Auswahl der momentan verfügbaren Technologien genannt und erklärt werden.

Die Erarbeitung von Kriterien für die Bewertung und Vergleich der Technologien in Kapitel~\myref{sec:kriterien} wird von einer Befragung unterstützt, die in \textit{Anhang} beschrieben ist. Hiermit schließt der theoretische Teil dieser Arbeit. 

In den Kapiteln~\myref{sec:einsatzorte} und \myref{sec:einsatzzwecke} werden dann mögliche Quell– bzw. Zielmedien genannt, bevor  die relevanten Schnittstellen in Kapitel~\myref{sec:bewertung} bewertet und miteinander verglichen werden.

Hieraus werden Kapitel~\myref{sec:handlungsempfehlungen} Handlungsempfehlungen bezüglich dem Einsatz der Technologien abgeleitet.
% (end)
\subsection{Abgrenzung} %(fold)

Da der Schritt von der persönlichen  zur unpersönlichen Kommunikation getan wird, sollte untersucht werden, für welche Zielsetzungen dies in Betracht kommt und wünschenswert ist. Dies wird in der vorliegenden Arbeit im Kapitel~\myref{sec:medienbruch} getan. Jedoch würde eine umfassende Behandlung dieses Themenkomplexes den Umfang dieser Arbeit sprengen und sollte somit gesondert durchgeführt werden.

Aus dem gleichen Grund wird auch auf die Gestaltungsmöglichkeiten und auf die Usability beim Einsatz der unterschiedlichen Technologien nur so weit eingegangen, wie es zur Beurteilung notwendig ist.

Die Technologien Nachrichtendienste und Chatsysteme sowie und Hotel––App werden nicht erschöpfend betrachtet werden, da sie nicht die erste Stufe der Kontaktaufnahme abdecken. Der Kunde muss zunächst durch eine andere Methode auf sie aufmerksam gemacht werden.
% (end)
