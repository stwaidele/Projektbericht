%!TEX root = /Users/stwaidele/Dropbox (Leisinger)/02 - AKAD/Projektbericht/Möglichkeiten der Digitalen Kontaktaufnahme im Endkundenbereich/vorlage.tex

\section{Einsatzorte} % (fold)
\label{sec:einsatzorte}

\subsection{An Stationen des Customer Journey} % (fold)
\label{sub:stationen_des_customer_journey}
Es sind die Phasen der Customer Journey, bei denen der Gast direkten Kontakt hat, während denen gastronomische Betriebe die digitale Kontaktaufnahme beeinflussen können. Dies beginnt bei der Buchung, ist in der Erlebnisphase am ausgeprägtesten und reicht bis zur Nachbereitung.

Während des Aufenthalts können entsprechende Hinweise auf Drucksachen und Schildern gedruckt werden um den Gast auf die digitalen Angebote des Betriebs zu führen. Dies kann an jedem \ac{POI}, also an Sehenswürdigkeiten jeglicher Art wie zum Beispiel an einem Gemälde, am Speisekartenaushang, an der Eingangstür oder im Rosengarten sein.

Bei Buchung und Nachbereitung sind diese Hinweise in der Korrespondenz zu platzieren.\footnote{Im Falle von E–Mails wird somit ein Wechsel von persönlicher zu unpersönilicher elektronischer Kommunikation angestrebt.}
% subsection stationen_des_customer_journey (end)

\subsection{In der Gastronomie} % (fold)
\label{sub:gastronomie}
In gastronomischen Betrieben lässt sich der Aufenthalt bzw. die Erlebnisphase der Customer Journey vereinfacht in fünf Phasen einteilen: Vor dem Besuch, vor der Bestellung, nach der Bestellung, während des Verzehrs\footnote{Bei mehrgängigen Menüs durch wartezeiten auf den nächsten Gang unterbrochen} und nach dem Besuch. Die Phasen eignen sich unterschiedlich gut für die digitale Kontaktaufnahme.  Vor der Bestellung und während des Verzehrs sind Verzögerungen durch den Gebrauch von Smartphones nicht wünschenswert.\footnote{Auch wenn die Behauptungen, dass der Service in einem New Yorker Restaurant sich durch Smartphonenutzung um durchschnittlich 50 Minuten verzögert wird (vgl. \cite{craiglist:slow}) nicht belegt sind, sind die geschilderten Phänomene zumindest mit kürzeren Zeitangaben doch plausibel. (vgl. \cite{craiglist:fake})} Vor dem Besuch, nach der Bestellung sowie nach den Besuch entstehen hierdurch keine Probleme, sondern für Gast und Betrieb vorteilhafte Anwendungsfälle. 

Der Gast kann durch entsprechende Hinweise am Speisekartenaushang, im Eingangsbereich, auf der Speisekarte, auf Tischaufstellern oder auf der Rechnung zu den digitalen Angeboten des Betriebs geleitet werden.
% subsection gastronomie (end)

\subsection{In der Hotellerie} % (fold)
\label{sub:hotellerie}
In Beherbergungsbetrieben hält sich der Gast während des Aufenthalts zwischen Anreise (Check–In) und Abreise (Check–Out) an unterschiedlichen Orten innerhalb des Betriebs auf. Für die Einnahme der Mahlzeiten gilt das im Abschnit~\myref{sub:gastronomie} gesagte. Weitere mögliche Orte der digitalen Kontktaufnahme sind das Zimmer des Gastes, öffentliche Bereiche wie Flure, Aufenthaltsräume und Wellnessbereich. Des weiteren können solche Angebote besondern an Orten, an denen viele Gäste warten gerne angenommen werden, z.B. in Bereich der Aufzüge oder in der Lobby.

% subsection hotellerie (end)


% section einsatzorte (end)

\newpage
\section{Einsatzzwecke} % (fold)
\label{sec:einsatzzwecke}

\subsection{Kunde mit Information versorgen} % (fold)
\label{sub:kunde_mit_information_versorgen}
VCard, Informationen zum momentanen Ort im Betrieb (Speisesaal--Öffnungszeiten, Bildbeschreibung, ...)

% subsection kunde_mit_information_versorgen (end)

\subsection{Kunde auf eine Website führen} % (fold)
\label{sub:kunde_auf_eine_website_fuhren}
Website des Betriebs, Facebook--Seite, Gewinnspiel, etc.
Auch: Eigene App
% subsection kunde_auf_eine_website_fuhren (end)

\subsection{Social Media Aktivitäten fördern} % (fold)
\label{sub:social_media_aktivitaten_fordern}
Reiseanalyse 2014, S5, unten! und http://www.fur.de/ra/news-daten/aktueller-newsletter/nl-0714-die-ra-customer-journey/ 
% subsection social_media_aktivitaten_fordern (end)

\subsection{Bewertungen einsammeln} % (fold)
\label{sub:bewertungen_einsammeln}
Spezialfall der Umleitung auf eine Website

% subsection bewertungen_einsammeln (end)

\subsection{Kontaktinformationen abfragen} % (fold)
\label{sub:kontaktinformationen_abfragen}
Spezialfall der Umleitung auf eine Website
% subsection kontaktinformationen_abfragen (end)

% section einsatzzwecke (end)

\newpage
\section{Bewertung der Technologien} % (fold)
\label{sec:bewertung}

\subsection{Bewertung QR Codes} % (fold)
\label{sub:bewertung_qr_codes}
Die maximale Speicherkapazität ist jedoch nur theoretisch erreichbar. Je größer die zu speichernde Datenmenge, desto feiner werden bei gleich bleibender Abbildungsgröße die zu scannenden Strukturen. Die Abbildungsgröße ist jedoch aufgrund der Nutzung der Smartphone–Kamera limitiert.\footnote{Auch bei extrem groß dargestellten QR–Codes muss dieser durch Anpassung des Abstandes dieser so verkleinert werden, dass er ganz von der Kamera erfasst wird.} So war keines der getesteten Smartphones\footnote{Apple iPhone 5 sowie Samsung Galaxy S5 mini} in der Lage, einen V40/L QR–Code zu erkennen. Auch bei kleineren Datenmengen treten bereits deutliche Unterschiede in der Erkennungsleistung sowohl zwischen den Geräten als auch zwischen den verwendeten Apps auf.

\todo{
\url{http://www.nielsen.com/de/de/insights/presseseite/2012/nielsen-das-smartphone-als-shopping-companion.html} \\
\url{http://www.springerprofessional.de/qr-codes-sind-ein-nerd--und-nischengeschichte/4781832.html} \\
\url{http://de.statista.com/statistik/daten/studie/237259/umfrage/bekanntheit-des-begriffes-qr-code-in-deutschland/} \\
\url{http://qrcode.wilkohartz.de/} \\
\url{http://www.saxoprint.de/blog/das-eigene-logo-in-einen-qr-code-einbinden/} \\ 
\url{http://stackoverflow.com/questions/5446421/encode-algorithm-qr-code} \\
\url{http://mojiq.kazina.com/} \\
\url{}
}
% subsection bewertung_qr_codes (end)


\subsection{Bewertung NFC–Tags} % (fold)
\label{sub:bewertung_nfc_tags}


% subsection bewertung_nfc_tags (end)

\subsection{Bewertung Bluetooth} % (fold)
\label{sub:bewertung_bluetooth}

% subsection bewertung_bluetooth (end)

\subsection{Bewertung Bluetooth Low Energie} % (fold)
\label{sub:bewertung_bluetooth_low_energie}

% subsection bewertung_bluetooth_low_energie (end)

\subsection{Bewertung Kurzlinks} % (fold)
\label{sub:bewertung_kurzlinks}

% subsection bewertung_kurzlinks (end)

% section bewertung_der_technologien (end)

\newpage
\section{Handlungsempfehlungen} % (fold)
\label{sec:handlungsempfehlungen}

\subsection{Technologische Handlungsempfehlungen} % (fold)
\label{sub:technologische_handlungsempfehlungen}

% subsection technologische_handlungsempfehlungen (end)

\subsection{Psychologische Handlungsempfehlungen} % (fold)
\label{sub:psychologische_handlungsempfehlungen}
Beim Einsatz von digitaler Komunikation und automatischen Bereitstellung von digitalen Informationen für den Kunden entsteht ein Interessenskonflikt: Einerseits soll das digitale Angebot attraktiv sein, um einen Nuzungsanreiz zu schaffen. Andererseits muss auch sichergestellt werden, dass allen Kunden ohne Nutzung weiterer Hilfsmittel trotzdem alle wichtigen und notwendigen Informationen zugänglich sind.\\
Hierbei ist es wichtig, dass dies nicht nur nach objektiven Maßstäben geschieht, sondern dass die Gefahr, dass sich Gäste benachteiligt fühlen minimiert wird. Dies könnte dadurch erreicht werden, dass zusätzlich zu den direkt zugänglichen Grundinformationen auf ein erweitertes Angebot hingewiesen wird, dass digital oder auch persönlich beim Personal erhältlich ist.

Des weiteren ist Abzuwägen, ob es in jeder möglichen Ausgangssituation wünschenswert ist, die Aufmerksamkeit des Gastes auf ein digitales Angebot zu lenken. So kann die Smartphonenutzung kurz vor der Bestellung den Geschäftsprozess deutlich verlangsamen, oder das gewünschte Ambiente stören. 

% subsection psychologische_handlungsempfehlungen (end)

% section handlungsempfehlungen (end)