%!TEX root = /Users/stwaidele/Dropbox (Leisinger)/02 - AKAD/Projektbericht/Möglichkeiten der Digitalen Kontaktaufnahme im Endkundenbereich/vorlage.tex

\section{Einsatzorte} % (fold)
\label{sec:einsatzorte}

\subsection{Am POI} % (fold)
\label{sub:am_poi}
Wo der Kunde die Information haben möchte.
% subsection am_poi (end)

\subsection{Im Lift} % (fold)
\label{sub:im_lift}
...oder in der Warteschlange, oder wo es sonst dem Kunde langweilig wird.
% subsection im_lift (end)

\subsection{Im Hotelzimmer} % (fold)
\label{sub:im_hotelzimmer}
Wo der Gast Zeit hat
% subsection im_hotelzimmer (end)

\subsection{Auf der Rechnung} % (fold)
\label{sub:auf_der_rechnung}
Nach der Reise ist vor der Reise
% subsection auf_der_rechnung (end)

\subsection{In der Rezeption} % (fold)
\label{sub:in_der_rezeption}
Hier ist eine Situationsbezogene Betrachtung notwendig: Beim Check--In und Check--Out hat der Gast wenig Zeit. Außerdem besteht der direkte Kontakt zu den Mitarbeitern. Hier ist von einem Einsatz der Technologien abzusehen.

Aber: In der Lounge oder Sitzgruppe ist ein Einsatz sinnvoll denkbar.
% subsection in_der_rezeption (end)

% section einsatzorte (end)

\newpage
\section{Einsatzzwecke} % (fold)
\label{sec:einsatzzwecke}

\subsection{Kunde mit Information versorgen} % (fold)
\label{sub:kunde_mit_information_versorgen}
VCard, Informationen zum momentanen Ort im Betrieb (Speisesaal--Öffnungszeiten, Bildbeschreibung, ...)

% subsection kunde_mit_information_versorgen (end)

\subsection{Kunde auf eine Website führen} % (fold)
\label{sub:kunde_auf_eine_website_fuhren}
Website des Betriebs, Facebook--Seite, Gewinnspiel, etc.
Auch: Eigene App
% subsection kunde_auf_eine_website_fuhren (end)

\subsection{Social Media Aktivitäten fördern} % (fold)
\label{sub:social_media_aktivitaten_fordern}
Reiseanalyse 2014, S5, unten! und http://www.fur.de/ra/news-daten/aktueller-newsletter/nl-0714-die-ra-customer-journey/ 
% subsection social_media_aktivitaten_fordern (end)

\subsection{Bewertungen einsammeln} % (fold)
\label{sub:bewertungen_einsammeln}
Spezialfall der Umleitung auf eine Website

% subsection bewertungen_einsammeln (end)

\subsection{Kontaktinformationen abfragen} % (fold)
\label{sub:kontaktinformationen_abfragen}
Spezialfall der Umleitung auf eine Website
% subsection kontaktinformationen_abfragen (end)

% section einsatzzwecke (end)

\newpage
\section{Bewertung der Technologien} % (fold)
\label{sec:bewertung}

\subsection{Bewertung QR Codes} % (fold)
\label{sub:bewertung_qr_codes}
Die maximale Speicherkapazität ist jedoch nur theoretisch erreichbar. Je größer die zu speichernde Datenmenge, desto feiner werden bei gleich bleibender Abbildungsgröße die zu scannenden Strukturen. Die Abbildungsgröße ist jedoch aufgrund der Nutzung der Smartphone–Kamera limitiert.\footnote{Auch bei extrem groß dargestellten QR–Codes muss dieser durch Anpassung des Abstandes dieser so verkleinert werden, dass er ganz von der Kamera erfasst wird.} So war keines der getesteten Smartphones\footnote{Apple iPhone 5 sowie Samsung Galaxy S5 mini} in der Lage, einen V40/L QR–Code zu erkennen. Auch bei kleineren Datenmengen treten bereits deutliche Unterschiede in der Erkennungsleistung sowohl zwischen den Geräten als auch zwischen den verwendeten Apps auf.

\todo{
\url{http://www.nielsen.com/de/de/insights/presseseite/2012/nielsen-das-smartphone-als-shopping-companion.html} \\
\url{http://www.springerprofessional.de/qr-codes-sind-ein-nerd--und-nischengeschichte/4781832.html} \\
\url{http://de.statista.com/statistik/daten/studie/237259/umfrage/bekanntheit-des-begriffes-qr-code-in-deutschland/} \\
\url{http://qrcode.wilkohartz.de/} \\
\url{http://www.saxoprint.de/blog/das-eigene-logo-in-einen-qr-code-einbinden/} \\ 
\url{http://stackoverflow.com/questions/5446421/encode-algorithm-qr-code} \\
\url{http://mojiq.kazina.com/} \\
\url{}
}
% subsection bewertung_qr_codes (end)


\subsection{Bewertung NFC–Tags} % (fold)
\label{sub:bewertung_nfc_tags}


% subsection bewertung_nfc_tags (end)

\subsection{Bewertung Bluetooth} % (fold)
\label{sub:bewertung_bluetooth}

% subsection bewertung_bluetooth (end)

\subsection{Bewertung Bluetooth Low Energie} % (fold)
\label{sub:bewertung_bluetooth_low_energie}

% subsection bewertung_bluetooth_low_energie (end)

\subsection{Bewertung Kurzlinks} % (fold)
\label{sub:bewertung_kurzlinks}

% subsection bewertung_kurzlinks (end)

% section bewertung_der_technologien (end)

\newpage
\section{Handlungsempfehlungen} % (fold)
\label{sec:handlungsempfehlungen}

% section handlungsempfehlungen (end)