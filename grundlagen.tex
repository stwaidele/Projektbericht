\section{Grundlagen}
\label{sec:grundlagen}

\subsection{Digitale Kontaktaufnahme}

Der Begriff \buzz{digitale Kontaktaufnahme} beschreibt im Rahmen dieser Arbeit den Vorgang der ersten Interaktion, die zwischen einem Kunden und einem Unternehmen mit Hilfe von digitalen Medien statt findet. 

Die digitale Kontaktaufnahme stellt daher einen \buzz{Medienbruch} dar, welcher im beidseitigen Interesse möglichst einfach und reibungslos für den Kunden verlaufen sollte. Für das Unternehmen ist es weiterhin von hohem Interesse, dass der Wechsel des Kontaktmediums möglichst zielgerichtet und treffsicher auf das digitale Angebot des Betriebs leitet. Nur somit können Streuverluste bzw. verlustbehaftete Umwege über Drittanbieter\footnote{etwa Buchungsplattformen (Buchungsprovisionen) oder Suchmaschinen (Anzeige von Konkurrenten)} vermieden werden.

Die digitale Kontaktaufnahme ist somit Startpunkt von Prozessen oder Teilprozessen, die mit Hilfe von digitalen Kommunikationsmitteln durchgeführt werden.

\subsection{Kunden}

Für die Zwecke dieser Untersuchung soll ein sehr weit gefasster Kundenbegriff zum Einsatz kommen, der Interessenten und potentielle Interessenten einschließt. Jedoch soll die Voraussetzung gelten, dass bereits ein nicht--digitaler Kontakt mit dem Unternehmen zu Stande gekommen ist. Dies kann durch den tatsächlichen Besuch des Betriebs, aber auch durch Prospekte, Plakate oder Informationstafeln geschehen sein.

\subsection{Notwendigkeit des Medienbruchs}
\label{sec:medienbruch}

Bei Betrachtung eines vor Ort anwesenden Kundens stellt der Wechsel des Kommunikationskanals auf ein digitales Medium einen teilweisen Rückschritt dar. Der unmittelbare Kontakt zum Servicemitarbeiter geht zunächst verloren und wird zumindest durch mittelbare Kommunikation und teilweise sogar durch die Wiedergabe unpersönlich hinterlegter Informationen ersetzt.
Betrachtet man jedoch die vielen Situationen, in denen ein Gast zwar vor Ort ist, aber keinen direkten Kontakt zum Servicepersonal hat, so können digital hinterlegte bzw. dynamisch generierte Informationen dem Gast einen deutlichen Mehrwert gegenüber gedruckten Materialien bieten.

Im Jahr 2013 haben 36\% der Reisenden mit mobilem Internetzugang während ihrer Urlaubsreise im Web 2.0 Beiträge veröffentlicht.\footnote{vgl. \cite{reiseanalyse}, Seite 5} Der Urlauber ist somit „sowohl Nutzer als auch Quelle von Informationen“.\footnote{\cite{reiseanalyse}, Seite 5} Daher ist es für touristische Betriebe wünschenswert, diesen Informationsfluss positiv zu beeinflussen. Der zumindest teilweise Wechsel des Kommunikationskanals schafft die Voraussetzungen dafür.

\section{Tourismus und Gastgewerbe}
\label{sec:tourismusbranche}

\subsection{Begriffsbestimmung}
Der Tourismus beschäftigt sich mit Erscheinungen und Beziehungen, die mit dem Verlassenen des üblichen Lebensmittelpunkt und dem vorübergehenden Aufenthalt in einer anderen Region verbunden sind.\footnote{vgl. \cite{gabler:tourismus}}

Das Gastgewerbe ist die Untermenge der touristischen Betriebe, welche sich mit der Unterbringung und Verpflegung von Reisenden befasst. Der Deutsche Hotel– und Gaststättenverband\acused{DeHoGa} (\ac{DeHoGa}) gliedert das Gastgewerbe in Hotellerie, sonstige Beherbergungsbetriebe, speisegeprägte Gastronomie, getränkegeprägte Gastronomie sowie Caterer.\footnote{vgl. \cite{dehoga:zahlenspiegel}, Seite 1} Eine grobe Einteilung ist somit in Beherbergung, d.h. die Möglichkeit des Übernachtens gegen Bezahlung für einen kurzen Zeitraum, und Gastronomie, d.h. das Angebot von Speisen und Getränken zum sofortigen Verzehr, möglich.\footnote{vgl. \cite{destatis:gastgewerbe}}

Speziell die Gäste der Getränke-- aber auch die Speisegastronomie fallen nicht unter die oben genannte Definition von Tourismus nach Gabler, da diese Betriebe durchaus auch am Ort des üblichen Lebensmittelpunkts besucht werden. Da diese Unterscheidung auch in den Veröffentlichungen des Statistischen Bundesamts und des \ac{DeHoGa}\footnote{z.B. \cite{dehoga:zahlenspiegel}} nicht zu tragen kommt, soll dies auch im Rahmen dieser Arbeit nicht berücksichtigt werden.

\subsection{Generelle Betrachtung des Gastgewerbes}

Wie schon aus den Definitionen hervorgeht, umfasst das Gastgewerbe ein Vielzahl von Betriebsarten, die sich in Größe, Betriebsstruktur, Qualitäts– sowie Preisniveau und Umsatzstärke deutlich von einander unterscheiden. Zum Gastgewerbe zählt die Imbissbude, die vom Inhaber alleine betrieben wird, das Sternerestaurant mit großer Küchen– und Servicebrigarde, der Wohnmobilstellplatz am Stadtrand und auch das 5–Sterne Grandhotel in bester Lage.

Der durchschnittliche Jahresumsatz der 224.309 gastgewerblichen Betriebe betrug 2012 etwas mehr als 310.000€.\footnote{vgl. Umsatzsteuerstatistik des Statistischen Bundesamt, zitiert nach \cite{dehoga:zahlenspiegel}} Da hierbei jedoch auch Betriebe mit sehr hohen Umsätzen enthalten sind\footnote{z.B. die 1440 Hotels der 50 größten Hotelketten mit über 5Mio. Euro pro Betrieb, (Eigene Berechnung auf Basis von \cite{ahgz:hotelier})}, kann davon ausgegangen werden, dass eine große Zahl von Betrieben deutlich unter 25.000€ pro Monat umsetzt.

Eine Gemeinsamkeit verbindet jedoch diese Betriebstypen: Die Mitarbeiter sind in der Regel wenig technikaffin. Für die Leistungserstellung wird zwar Technik benötigt, diese wird jedoch vom Personal lediglich bedient und nicht eingerichtet.\footnote{z.B. Buchungssysteme, Registrierkassen oder auch programmierbare Öfen} Die Servicemitarbeiter sind im direkten Kundenkontakt begabt und geschult, die Küchenmitarbeiter hingegen erstellen ihre Leistung meist ohne Kundenkontakt. Bei beiden Gruppen gehören Fähigkeiten der digitalen Kommunikation mit den Kunden nicht zum Anforderungsprofil.\\
Eine Ausnahme bilden hier die Mitarbeiter der Hotelrezeption bzw. Bankettabteilung von entsprechenden Betrieben. In diesen Bereichen ist die schriftliche Kommunikation mit Interessenten und Kunden durchaus üblich, welche heute überwiegend per E–Mail abgewickelt wird. 

Spezialisierte Mitarbeiter in den Bereichen IT oder Marketing findet man erst in Betrieben von deutlich überdurchschnittlicher Größe. 

\subsection{Kleine Betriebe im Gastgewerbe}

Im Rahmen dieser Arbeit sollen diejenigen Betriebe als „klein“ angesehen werden, bei denen wie im vorhergehenden Abschnitt beschrieben die Aufgaben der IT bzw. des Marketings nicht von spezialisierten Mitarbeitern übernommen werden. Arbeiten, die besonderes Know-How verlangen werden in den kleinen Betrieben aus Dienstleister vergeben.\footnote{z.B. Website–Erstellung, Installation von Gäste–WLAN, betriebliche EDV, Prospektdesign, Entwurf einer Vorlage für Mailings, usw.} Die eingerichteten Systeme werden dann von Mitarbeitern neben deren gastronomischer Hauptaufgabe gepflegt.\footnote{z.B. Aktualisierung der Website, Pflege des Buchungssystems, Zusammenstellung von Ausflugszielen, verfassen von Mailings}

Je mehr technisches Know–How bei den Mitarbeitern nicht nur vorhanden ist, sondern auch offiziell laut Stellenbeschreibung gefordert und die entsprechenden Tätigkeiten zur Hauptaufgabe des Mitarbeiters gehört, desto eher ist der Betrieb als „mittelgroß“ oder bei Vorhandensein von entsprechenden Abteilungen für Marketing und/oder IT als „groß” zu bezeichnen.

\section{Technologische Möglichkeiten der digitalen Kontaktaufnahme}
\label{sec:technologien}

\subsection{QR–Codes}
\todo{
\url{http://www.nielsen.com/de/de/insights/presseseite/2012/nielsen-das-smartphone-als-shopping-companion.html} \\
\url{http://www.springerprofessional.de/qr-codes-sind-ein-nerd--und-nischengeschichte/4781832.html} \\
\url{http://de.statista.com/statistik/daten/studie/237259/umfrage/bekanntheit-des-begriffes-qr-code-in-deutschland/} \\
\url{http://qrcode.wilkohartz.de/} \\
\url{http://www.saxoprint.de/blog/das-eigene-logo-in-einen-qr-code-einbinden/} \\ 
\url{http://stackoverflow.com/questions/5446421/encode-algorithm-qr-code} \\
\url{http://mojiq.kazina.com/} \\
\url{}
}

\subsection{NFC--Tags}
\todo{
\url{http://www.androidmag.de/tipps/einsteiger/den-alltag-meistern-mit-nfc-tags/} \\
\url{http://www.usabilityblog.de/2011/11/near-field-communication-nfc-im-tourismus-wie-beim-reisen-die-on-und-offlinewelt-nutzerfreundlich-verbunden-werden/} \\
\url{http://www.pcwelt.de/ratgeber/Ratgeber-Software-NFC-Tags-selber-machen-7769827.html} \\

}
\subsection{Bluetooth}
\subsection{Bluetooth Low Energy}
z.B. iBeacon

\todo{
\url{http://t3n.de/news/apple-ibeacon-nfc-499992/} \\
\url{http://www.huffingtonpost.de/christian-eggert/wie-ibeacon-unser-leben-v_b_5436171.html}\\
\url{http://de.wikipedia.org/wiki/IBeacon} }

\subsection{Kurzlinks}
\subsection{Internet--Nachrichtendienste}

z.B. Twitter für nicht an eine Mobilfunknummer gebundene Nachrichtendienste. Auch: iMessage, evt. Facebook Messenger

Aufgrund der hohen Anforderung bzgl. syncroner Kommunikation ist diese Möglichkeit nicht optimal für kleine Betriebe und wird nur im Grundlagenteil behandelt.

\subsection{Mobilfunk--Nachrichtendienste}

z.B. WhatsApp für Nachrichtendienste, die an eine Mobilfunknummer gebunden sind, selbst wenn die eigentliche Kommunikation über Internettechniken abgewickelt wird. Auch: SMS.

Aufgrund der hohen Anforderung bzgl. synchroner Kommunikation ist diese Möglichkeit nicht optimal für kleine Betriebe und wird nur im Grundlagenteil behandelt.

\todo{
Messenger statt FB:
\url{http://www.fur.de/ra/news-daten/aktueller-newsletter/nl-0714-die-ra-customer-journey/}
bzw.
\url{http://www.fur.de/fileadmin/user_upload/Newsletter/Newsletter_2014Juli/RA_Newsletter_07-2014_CustomerJourney.pdf}

\url{http://blog.trendone.com/wp-content/uploads/2009/11/Pressemeldung-Outernet-White-Paper-TrendONE-November-2009.pdf} \\
\url{http://www.usabilityblog.de/blog-autoren/eric-horster/}
}


\subsection{Hotel--Apps}

z.B. Protel „Voyager“

Aufgrund der hohen Anforderung bzgl. Kapitaleinsatz ist diese Möglichkeit nicht optimal für kleine Betriebe und wird nur im Grundlagenteil behandelt.

\section{Bewertungskriterien}
\label{sec:kriterien}

Um die verschiedenen technischen Möglichkeiten miteinander vergleichen zu können, bieten sich folgende Kriterien an:

\begin{itemize}
\item Benötigte Hard-- und Software auf Seiten des Hotels
\item Benötigte Hard-- und Software auf Seiten des Kunden
\item Bekanntheitsgrad 
\item Verbreitungsgrad 
\item Benutzerfreundlichkeit
\item Technische Möglichkeiten\footnote{z.B. Speicherkapazität}
\item Finanzielle Anforderungen
\item Personelle Anforderungen
\item Mißbrauchsmöglichkeiten\footnote{z.B. überklebte QR--Codes}
\end{itemize}