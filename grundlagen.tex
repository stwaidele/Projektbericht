\section{Grundlagen}
\label{sec:grundlagen}

\subsection{Digitale Kontaktaufnahme}

Der Begriff \buzz{digitale Kontaktaufnahme} beschreibt im Rahmen dieser Arbeit den Vorgang der ersten Interaktion, die zwischen einem Kunden und einem Unternehmen mit Hilfe von digitalen Medien statt findet. 

Die digitale Kontaktaufnahme stellt daher einen \buzz{Medienbruch} dar, welcher im beidseitigen Interesse möglichst einfach und reibungslos für den Kunden verlaufen sollte. Für das Unternehmen ist es weiterhin von hohem Interesse, dass der Wechsel des Kontaktmediums möglichst zielgerichtet und treffsicher auf das digitale Angebot des Betriebs leitet. Nur somit können Streuverluste bzw. verlustbehaftete Umwege über Drittanbieter\footnote{etwa Buchungsplattformen (Buchungsprovisionen) oder Suchmaschinen (Anzeige von Konkurrenten)} vermieden werden.

Die digitale Kontaktaufnahme ist somit Startpunkt von Prozessen oder Teilprozessen\footnote{\todo{Zitat, am besten von Scheer oder so}}, die mit Hilfe von digitalen Kommunikationsmitteln durchgeführt werden.

\subsection{Kunden}

Für die Zwecke dieser Untersuchung soll ein sehr weit gefasster Kundenbegriff zum Einsatz kommen, der Interessenten und potentielle Interessenten einschließt. Jedoch soll die Voraussetzung gelten, dass bereits ein nicht--digitaler Kontakt mit dem Unternehmen zu Stande gekommen ist. Dies kann durch den tatsächlichen Besuch des Betriebs, aber auch durch Prospekte, Plakate oder Informationstafeln geschehen sein.

\subsection{Gastgewerbe}

\todo{
\url{http://www.dehoga-bundesverband.de/daten-fakten-trends/betriebsarten/}
\\ \url{http://www.dehoga-bundesverband.de/daten-fakten-trends/gastgewerbe-im-ueberblick/}
\\ \url{http://www.dehoga-bundesverband.de/daten-fakten-trends/umsatzentwicklungen/}
\\ \url{http://www.dehoga-bundesverband.de/fileadmin/Inhaltsbilder/Daten_Fakten_Trends/Zahlespiegel_und_Branchenberichte/Zahlenspiegel/Zahlenspiegel_1__Quartal_2014.pdf}
\\ \url{http://www.dehoga-bundesverband.de/presse/pressemitteilungen/gastgewerbe-2013-solides-umsatzplus-von-12-prozent-viertes-jahr-mit-umsatzplus-in-folge-2014-02-17-1000/}
}


\subsection{Notwendigkeit des Medienbruchs}
\label{sec:medienbruch}

Bei Betrachtung eines vor Ort anwesenden Kundens stellt der Wechsel des Kommunikationskanals auf ein digitales Medium einen teilweisen Rückschritt dar. Der unmittelbare Kontakt zum Servicemitarbeiter geht zunächst verloren und wird zumindest durch mittelbare Kommunikation und teilweise sogar durch die Wiedergabe unpersönlich hinterlegter Informationen ersetzt.
Betrachtet man jedoch die vielen Situationen, in denen ein Gast zwar vor Ort ist, aber keinen direkten Kontakt zum Servicepersonal hat, so können digital hinterlegte bzw. dynamisch generierte Informationen dem Gast einen deutlichen Mehrwert gegenüber gedruckten Materialien bieten.\footnote{Zitat: Beleg!} 

Im Jahr 2013 haben 36\% der Reisenden mit mobilem Internetzugang während ihrer Urlaubsreise im Web 2.0 Beiträge veröffentlicht.\footnote{vgl. \cite{reiseanalyse}, Seite 5} Der Urlauber ist somit „sowohl Nutzer als auch Quelle von Informationen“.\footnote{\cite{reiseanalyse}, Seite 5} Daher ist es für touristische Betriebe wünschenswert, diesen Informationsfluss positiv zu beeinflussen. Der zumindest teilweise Wechsel des Kommunikationskanals schafft die Voraussetzungen dafür.

\todo{
Messenger statt FB:
http://www.fur.de/ra/news-daten/aktueller-newsletter/nl-0714-die-ra-customer-journey/ 
bzw.
http://www.fur.de/fileadmin/user_upload/Newsletter/Newsletter_2014Juli/RA_Newsletter_07-2014_CustomerJourney.pdf

\url{http://blog.trendone.com/wp-content/uploads/2009/11/Pressemeldung-Outernet-White-Paper-TrendONE-November-2009.pdf} \\
\url{http://www.usabilityblog.de/blog-autoren/eric-horster/}\\

}
\section{Technologische Möglichkeiten der digitalen Kontaktaufnahme}
\label{sec:technologien}

\subsection{QR–Codes}
\todo{
\url{http://www.nielsen.com/de/de/insights/presseseite/2012/nielsen-das-smartphone-als-shopping-companion.html} \\
\url{http://www.springerprofessional.de/qr-codes-sind-ein-nerd--und-nischengeschichte/4781832.html} \\
\url{http://de.statista.com/statistik/daten/studie/237259/umfrage/bekanntheit-des-begriffes-qr-code-in-deutschland/} \\
\url{http://qrcode.wilkohartz.de/} \\
\url{http://www.saxoprint.de/blog/das-eigene-logo-in-einen-qr-code-einbinden/} \\ 
\url{http://stackoverflow.com/questions/5446421/encode-algorithm-qr-code} \\
\url{http://mojiq.kazina.com/} \\
\url{}
}

\subsection{NFC--Tags}
\todo{
\url{http://www.androidmag.de/tipps/einsteiger/den-alltag-meistern-mit-nfc-tags/} \\
\url{http://www.usabilityblog.de/2011/11/near-field-communication-nfc-im-tourismus-wie-beim-reisen-die-on-und-offlinewelt-nutzerfreundlich-verbunden-werden/} \\
\url{http://www.pcwelt.de/ratgeber/Ratgeber-Software-NFC-Tags-selber-machen-7769827.html} \\

}
\subsection{Bluetooth}
\subsection{Bluetooth Low Energy}
z.B. iBeacon

\todo{
\url{http://t3n.de/news/apple-ibeacon-nfc-499992/} \\
\url{http://www.huffingtonpost.de/christian-eggert/wie-ibeacon-unser-leben-v_b_5436171.html}\\
\url{http://de.wikipedia.org/wiki/IBeacon} }

\subsection{Kurzlinks}
\subsection{Internet--Nachrichtendienste}

z.B. Twitter für nicht an eine Mobilfunknummer gebundene Nachrichtendienste. Auch: iMessage, evt. Facebook Messenger

Aufgrund der hohen Anforderung bzgl. syncroner Kommunikation ist diese Möglichkeit nicht optimal für kleine Betriebe und wird nur im Grundlagenteil behandelt.

\subsection{Mobilfunk--Nachrichtendienste}

z.B. WhatsApp für Nachrichtendienste, die an eine Mobilfunknummer gebunden sind, selbst wenn die eigentliche Kommunikation über Internettechniken abgewickelt wird. Auch: SMS.

Aufgrund der hohen Anforderung bzgl. synchroner Kommunikation ist diese Möglichkeit nicht optimal für kleine Betriebe und wird nur im Grundlagenteil behandelt.

\subsection{Hotel--Apps}

z.B. Protel „Voyager“

Aufgrund der hohen Anforderung bzgl. Kapitaleinsatz ist diese Möglichkeit nicht optimal für kleine Betriebe und wird nur im Grundlagenteil behandelt.

\section{Bewertungskriterien}
\label{sec:kriterien}

Um die verschiedenen technischen Möglichkeiten miteinander vergleichen zu können, bieten sich folgende Kriterien an:

\begin{itemize}
\item Benötigte Hard-- und Software auf Seiten des Hotels
\item Benötigte Hard-- und Software auf Seiten des Kunden
\item Bekanntheitsgrad 
\item Verbreitungsgrad 
\item Benutzerfreundlichkeit
\item Technische Möglichkeiten\footnote{z.B. Speicherkapazität}
\item Finanzielle Anforderungen
\item Personelle Anforderungen
\item Mißbrauchsmöglichkeiten\footnote{z.B. überklebte QR--Codes}
\end{itemize}